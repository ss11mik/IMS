\documentclass[11pt, a4paper]{article}


\usepackage[czech]{babel}
\usepackage[utf8]{inputenc}
\usepackage[T1]{fontenc}


\usepackage[hidelinks, unicode]{hyperref}
\usepackage[left=2cm,text={17cm, 24cm},top=3cm]{geometry}
\usepackage{graphicx}
\usepackage{float}
\graphicspath{{./imgs/}}



\begin{document}

\begin{titlepage}
\begin{center}
\Huge
\textsc{Vysoké učení technické v Brně}\\
\huge
\textsc{Fakulta informačních technologií}\\
\vspace{\stretch{0.382}}
\LARGE
{\bf SHO ve výrobě}\\Simulační studie\\IMS
\vspace{\stretch{0.618}}
\end{center}
\Large
\today\hfill
\begin{tabular}{l l l}
		David Podeszwa & (xpodes05)\\
		Ondřej Mikula & (xmikul69)\\
\end{tabular}
\end{titlepage}

\newpage
\tableofcontents
\newpage


\section{Úvod}
V této práci je popsán simulační model výroby hranatého potrubí ve společnosti KLMN spol s.r.o. Na základě modelu a simulačních experimentů budou ukázána slabá místa výroby. Cílem práce je  navrhnout vylepšení výrobního procesu a
%zjistit, zda je možné tato slabá místa vylepšit či odstranit a navrhnout další vylepšení výrobního procesu a
experimenty potvrdit či vyvrátit jejich účinnost.
\subsection{Autoři a zdroje informací}
Autory projektu jsou David Podeszwa a Ondřej Mikula. 


Informace byly čerpány z bakalářské práce \href{http://digilib.k.utb.cz/bitstream/handle/10563/22155/%20ih%C3%A1k_2012_bp.pdf?sequence=1}{Analýza současného výrobního procesu ve 
vybrané firmě}. TODO:%%TODO: Bibtext citace zdroje
\subsection{Ověření validity}
Konkrétní informace o výrobě byly čerpány z uvedené bakalářské práce, jejíž autor získal informace přímo od dané firmy, a tudíž by měly být přesné. Prvotní experiment na výchozím modelu ověřil, že průměrná doba výroby produktu se shoduje s dobou uvedenou ve zdroji.
\section{Rozbor tématu a použitých metod/technologií}
Tématem simulace je výroba hranatého potrubí. Samotný proces výroby je rozdělen do více částí. Materiál použitý k výrobě je pouze jeden, a to pozinkovaný plech. Výrobu zahajuje střihač, který na stroji nastříhá tento plech dle potřebných rozměrů.
\begin{table}[H]
    \centering
    \begin{tabular}{|l|l|l|l|}
     \hline  & \textbf{operace} &  \textbf{čas [min]} &  \textbf{přenos na další pracoviště [min]}\\ \hline
    1. &                Stříhání plechu   &  4 &  0,25\\ \hline
    2. &                Z-profil    &  1,5 &  0,25\\ \hline
    3. &                Ohýbání plechu    &  2  &  0,5\\ \hline
    4. &                Uzavírání profilu    & 2 &  1\\  \hline
    5. &                Řezání přírubových profilů    &  1 &  0\\ \hline
    6. &               Kompletace spojovacích přírub    &  3 &  0\\ \hline
    7. &                Osazení přírubami    &  2 &  1\\
    \hline
    8. &                Kontrola kvality    & 1  &  N/A\\
    \hline

    
    \end{tabular}
    \caption{Výrobní postup}
    \label{rgrg}
\end{table}

Po posledním kroku je trubka odnesena do skladu, kde čeká na expedici následující den, ale to již není součástí modelu, neboť ten se zabývá jen samotnou výrobou.

\subsection{Použité postupy}
Pro vytvoření simulačního modelu byl použit jazyk C++ a jeho standardní knihovny společně s knihovnou SIMLIB [TODO: citace, odkaz]. Ta obsahuje veškeré nástroje potřebné pro vytvoření simulačního modelu systému hromadné obsluhy [TODO: citace z prezentace]. 
\subsection{Původ použitých postupů}
\begin{itemize}
    \item Pro implementaci byl použit programovací jazyk C++ konkrétně kompilováno se standardem C++17. - \href{https://en.cppreference.com/w/cpp}{\texttt{https://en.cppreference.com/w/cpp}}
    \item Knihovna SIMLIB -  \href{http://www.fit.vutbr.cz/~peringer/SIMLIB/}{\texttt{http://www.fit.vutbr.cz/~peringer/SIMLIB/}}
    \item Nástroj GNU make. Použit k automatizaci kompilace -  \href{https://www.gnu.org/software/make/}{\texttt{https://www.gnu.org/software/make/}}
\end{itemize}

\section{Koncepce}




\section{Architektura simulačního modelu/simulátoru}
\subsection{Namapování veličin}
Simlib
10 týdnů
čas v minutách

\section{Podstata simulačních experimentů a jejich průběh}
Obecným cílem experimentů bylo naýšení produkce, a tím i zisku, simulované firmy.

\subsection{Postup experimentování}
Nejdříve byly identifikovány potenciální cesty pro zvýšení produktivity. Každé takové cestě odpovídá jeden proběhlý experiment. Některé byly převzaty z neověřených tezí v literatuře, další se zaměřily na identifikaci problémů či zapojení nových metod do postupu výroby.

V průběhu experimentu byly použity metody půlení intervalu a záměrného naddimenzování některých hodnot.

\subsection{Jednotlivé experimenty}

\subsubsection{Validita výchozího modelu}
První experiment byl proveden nad modelem, simulujícím reálný stav výroby. Porovnáním jeho výstupů s realitou bylo provedeno ověření. Vzhledem ke shodnosti průměrného času výroby jednoho potrubí a výrobní kapacity lze model považovat za validní a pokračovat k dalším experimentům. V tomto experimentu nedochází k zahlcování a jednotlivá pracoviště jsou zatížena přibližně na 20 až 70 \%.



\subsubsection{Úzká místa}
Následným experimentem byla identifikována úzká místa ve výrobě - tedy stanoviště, která brzdí výrobu a na která musí následující stanoviště čekat. Hlavními ukazateli byly vytíženost stanoviště, tedy kolik procent času bylo využito, a velikost fronty, která se před ním tvořila.

Ve stavu odpovídajícím realitě je nejvíce vytíženo 1. pracoviště (stříhání plechu), díky čemuž další stíhají a nikde nevznikají fronty. Prvním logickým krokem tedy bylo navýšit produkitivitu tohoto stanoviště.

Jednoduchá simulace zdvojení tohoto stanoviště (tedy zakoupení 2. stejného stroje a přijmutí dalšího zaměstnance, či práce na jednom stroji na 2 směny) ukázala, že před stanicí pro kompletaci (3.) a osazování (4.) by se začaly tvořit fronty (které s postupem času jen narůstají).

Metodou půlení intervalu bylo následně zjištěno, že při zachování plynulosti celé výroby lze efektivitu stříhání plechu navýšit až o 41 \% (při vyšších hodnotách se již začnou tvořit stále se zvětšující fronty). Tím se celková produkce navýší o odpovídajících 41 \%.

\subsubsection{Nevyužitá práce}
Další experiment se zaobírá opačným problémem - identifikací pracovišť, která jsou málo využita a experimentováním, jak tento negavitní jev eliminovat.

Ve výchozím modelu jsou pracoviště \uv{řezání přírubových profilů} a \uv{kontrola kvality} nejméně využita - na 24, resp. 29 \%. Při snížení jejich výrobní kapacity na polovinu jsou stále využity jen na 47 a 59 \%. Z toho vyplývá, že zaměstnanci na těchto stanovištích mohou pracovat na poloviční úvazek - za udržení produktivity výroby a navýšení průměrného času výroby jedné trubky na TODO minut.


\subsubsection{Navrhovaná úprava rozmístění}
V tomto experimentu bylo simulováno nové rozmístění jednotlivých stanic podle návrhu v TODO. To bylo provedeno přenastavením časů pro transport mezi stanicemi dle zdroje. Následná simulace tohoto experimentu potvrdila zrychlení výroby jednoho kusu z 20 minut na 18.85 minut, což odpovídá zdroji.


\subsubsection{Dávkové předávání}
Cílem tohoto experimentu bylo zvýšit produkci tím, že polotovary budou přenášeny mezi stanicemi až po nahromadění určitého počtu kusů (v součastnosti se přenáší po jednom). Experiment byl spuštěn několikrát s různými počty kusů, po kterých jsou přenášeny. Každé pracoviště má vlastní překladiště, předání se tedy provede po jeho zaplnění, nezávisle na ostatních částech linky.

Předpoklad, že průměrná doba pro vyrobení jednoho kusu se zvětší, se potvrdil.


\subsubsection{Vyhrazený zaměstnanec pro přenos mezi pracovišti}
Tento experiment simuluje přijetí nového zaměstnance, jehož pracovní náplní by bylo pouze přenášet polotovary mezi jednotlivými stanovišti.

Nový zaměstnanec se řídí pro zjednodušení následujícím algoritmem: Zjistí, u kterého stanoviště je nahromaděno nejvíce polotovarů, a ty přenese.

Simulace chodu výroby s jedním takovým zaměstnancem ukázala, že produkce firmy by se zvýšila asi o 6 \%. Nový zaměstananec by tvořil 1/13, tedy cca 8 \%, zaměstanců, ovšem potřeboval by prakticky nulovou kvalifikaci a tedy nižší mzdu, jeho zařazení by tudíž mohlo být pro firmu výhodné.

Vedlejším efektem zařazení pomocného pracovníka bylo zvýšení průměrné doby výroby jednoho produktu, a to o více než dvojnásobek. Tato veličina ovšem ovlivňuje pouze dobu náběhu ze zastavení do plné produkce.

%mnohem méně důležitá, než počet vyprodukovaných výrobků za dobu simulace.

V experimentu s více zaměstnanci pro přenášení se již produkce téměř nezvýšila, pouze se zanedbatelně snížil průměrný čas výroby jednoho produktu, stále však nedosáhl na hodnotu bez přenašečů.

\subsubsection{Kombinované experimenty}

\subsection{Závěry experimentů}
Z provedených experimentů vyplývá, že


\section{Shrnutí simulačních experimentů a závěr}



\end{document}
