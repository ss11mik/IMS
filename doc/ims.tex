\documentclass[11pt, a4paper]{article}


\usepackage[czech]{babel}
\usepackage[utf8]{inputenc}
\usepackage[T1]{fontenc}


\usepackage[hidelinks, unicode]{hyperref}
\usepackage[left=2cm,text={17cm, 24cm},top=3cm]{geometry}
\usepackage{graphicx}
\usepackage{float}
\graphicspath{{./imgs/}}



\begin{document}

\begin{titlepage}
\begin{center}
\Huge
\textsc{Vysoké učení technické v Brně}\\
\huge
\textsc{Fakulta informačních technologií}\\
\vspace{\stretch{0.382}}
\LARGE
{\bf SHO ve výrobě}\\Simulační studie\\IMS
\vspace{\stretch{0.618}}
\end{center}
\Large
\today\hfill
\begin{tabular}{l l l}
		David Podeszwa & (xpodes05)\\
		Ondřej Mikula & (xmikul69)\\
\end{tabular}
\end{titlepage}

\newpage
\tableofcontents
\newpage


\section{Úvod}
V této práci je popsán simulační model výroby hranatého potrubí ve společnosti KLMN spol s.r.o. Na základě modelu a simulačních experimentů budou ukázána slabá místa výroby. Smyslem práce je pak zjistit, zda-li je možné tyto slabá místa vylepšit nebo úplně odstranit. 
\subsection{Autoři a zdroje informací}
Autory projektu jsou David Podeszwa a Ondřej Mikula. 


Informace byly čerpány z bakalářské práce \href{http://digilib.k.utb.cz/bitstream/handle/10563/22155/%20ih%C3%A1k_2012_bp.pdf?sequence=1}{Analýza současného výrobního procesu ve 
vybrané firmě}. TODO:%%TODO: Bibtext citace zdroje
\subsection{Ověření validity}
Veškeré informace o výrobě byly čerpány z uvedené bakalářské práce, jejíž autor čerpal informace přímo od dané firmy, a tudíž by měly být přesné. Prvotní experiment na výchozím modelu pak ověřil, že průměrná doba výroby produktu trvá tak dlouho jak uvádí zdroj.
\section{Rozbor tématu a použitých metod/technologií}
Tématem práce je simulace výroby hranatého potrubí. Samotný proces výroby je rozdělen do více částí. Materiálem použitý k výrobě je pouze jeden a to pozinkovaný plech. Výrobu zahajuje stříhač, ktery na stroji nastříhá tento plech dle rozměrů.
\begin{table}[H]
    \centering
    \begin{tabular}{|l|l|l|}
     \hline  & \textbf{Operace} &  \textbf{Čas (min)}\\ \hline
    1. &                Stříhání plechu   &  4\\ 
    2. &\hspace{0.1cm}  Transport   &  0,25\\ \hline
    3. &                Válcování    &  4\\ 
    4. &\hspace{0.1cm}  Transport   &  0,25\\ \hline
    5. &                Ohýbání plechu    &  4\\ 
    6. &\hspace{0.1cm}  Transport   &  0,25\\ \hline
    7. &                Uzavírání profilu    &  4\\ 
    8. &\hspace{0.1cm}  Transport   &  0,25\\ \hline
    9. &                Řezání přírubových profilů    &  4\\ \hline
    10. &               Kompletace spojovacích přírub    &  4\\ \hline
    12. &                Osazení přírubami    &  4\\ 
    13. &\hspace{0.1cm}  Transport   &  0,25\\ \hline
    
    \end{tabular}
    \caption{Kapce}
    \label{rgrg}
\end{table}

Po posledním kroku je trubka odnesena na sklad kde čeká na expedici následující den, ale to už není součástí modelu, neboť ten se zabývá jen samotnou výrobou.

\subsection{Význačné zdroje}

\subsection{Prostředí}

\section{Simulace}
Simlib
10 týdnů
čas v minutách

\section{Experimenty}
Obecným cílem experimentů bylo naýšení produkce, a tím i zisku, simulované firmy.

\subsection{Postup experimentování}
Nejdříve byly identifikovány potenciální cesty pro zvýšení produktivity. Každé takové cestě odpovídá jeden proběhlý experiment. Některé byly převzaty z neověřených tezí v literatuře, další se zaměřily na identifikaci problémů či zapojení nových metod do postupu výroby.

V průběhu experimentu byly použity metody půlení intervalu a záměrného naddimenzování některých hodnot.

\subsection{Jednotlivé experimenty}

%\subsubsection{Úzká místa}

\subsubsection{Úzká místa}
Úvodním experimentem byla identifikována úzká místa ve výrobě - tedy stanoviště, která brzdí výrobu a na která musí následující stanoviště čekat. Hlavními ukazateli byly vytíženost stanoviště, tedy kolik procent času bylo využito, a velikost fronty, která se před ním tvořila.

Ve stavu odpovídajícím realitě je nejvíce vytíženo 1. pracoviště (stříhání plechu), díky čemuž další stíhají a nikde nevznikají fronty. Prvním logickým krokem tedy bylo navýšit produkitivitu tohoto stanoviště.

Jednoduchá simulace zdvojení tohoto stanoviště (tedy zakoupení 2. stejného stroje a přijmutí dalšího zaměstnance, či práce na jednom stroji na 2 směny) ukázala, že před stanicí pro kompletaci (3.) a osazování (4.) by se začaly tvořit fronty (které s postupem času jen narůstají).

Metodou půlení intervalu bylo následně zjištěno, že při zachování plynulosti celé výroby lze efektivitu stříhání plechu navýšit až o 41 \% (při vyšších hodnotách se již začnou tvořit stále se zvyšující fronty). Tím se celková produkce navýší o odpovídajících 41 \%.


\subsubsection{Navrhovaná úprava rozmístění}
V tomto experimentu bylo simulováno nové rozmístění jednotlivých stanic podle návrhu v TODO. To bylo provedeno přenastavením časů pro transport mezi stanicemi dle zdroje. Následná simulace tohoto experimentu potvrdila zrychlení výroby jednoho kusu z 20 minut na 18.85 minut.


\subsubsection{Dávkové předávání}
Cílem tohoto experimentu bylo zvýšit produkci tím, že polotovary budou přenášeny mezi stanicemi až po nahromadění určitého počtu kusů (v součastnosti se přenáší po jednom). Experiment byl spuštěn několikrát s různými počty kusů, po kterých jsou přenášeny. Každé pracoviště má vlastní překladiště, předání se tedy provede po jeho zaplnění, nezávisle na ostatních částech linky.

Předpoklad, že průměrná doba pro vyrobení jednoho kusu se zvětší, se potvrdil.


\subsubsection{Vyhrazený zaměstnanec pro přenos mezi pracovišti}
Tento experiment simuluje přijetí nového zaměstnance, jehož pracovní náplní by bylo pouze přenášet polotovary mezi jednotlivými stanovišti.

Simulace chodu výroby s jedním takovým zaměstnancem ukázala, že produkce firmy by se zvýšila asi o 6 \%. Nový zaměstananec by tvořil 1/13, tedy cca 8 \%, zaměstanců, ovšem potřeboval by prakticky nulovou kvalifikaci a tedy nižší plat, jeho zařazení by tudíž mohlo být pro firmu výhodné.

Vedlejším efektem zařazení pomocného pracovníka bylo zvýšení průměrné doby výroby jednoho produktu, a to o více než dvojnásobek. Tato veličina ovšem ovlivňuje pouze dobu náběhu ze zastavení do plné produkce.

%mnohem méně důležitá, než počet vyprodukovaných výrobků za dobu simulace.

V experimentu s více zaměstnanci pro přenášení se již produkce téměř nezvýšila, pouze se zanedbatelně snížil průměrný čas výroby jednoho produktu, stále však nedosáhl na hodnotu bez přenašečů.

\subsubsection{Kombinované experimenty}

\subsection{Závěry experimentů}
Z provedených experimentů vyplývá, že


\section{Závěr}

\end{document}
